%% LyX 2.2.3 created this file.  For more info, see http://www.lyx.org/.
%% Do not edit unless you really know what you are doing.
\documentclass[english]{article}
\usepackage[T1]{fontenc}
\usepackage[latin9]{inputenc}
\usepackage{geometry}
\geometry{verbose,tmargin=1.5in,bmargin=1.5in,lmargin=1.5in,rmargin=1.5in}
\usepackage{color}
\usepackage{babel}
\usepackage{amsmath}
\usepackage{babel}
\newcommand{\GeV}{\,{\rm GeV}}
\newcommand{\MHz}{\,{\rm MHz}}
\usepackage{listings}
\usepackage{float}
\usepackage{graphicx}
\graphicspath{{./plots/}}

\lstdefinestyle{customc}{
  belowcaptionskip=1\baselineskip,
  breaklines=true,
  frame=L,
  xleftmargin=\parindent,
  language=C,
  showstringspaces=false,
  basicstyle=\footnotesize\ttfamily,
  keywordstyle=\bfseries\color{green},
  commentstyle=\itshape\color{red},
  identifierstyle=,
  stringstyle=\color{blue},
}

\lstset{escapechar=@,style=customc}


\begin{document}

\title{Prelab 10 : Arduino}

\author{Omkar H. Ramachandran}
\maketitle

\section{Set up the Arduino software}
Done. Because I've got Arch Linux on my machine and none of the IDEs work
properly, I just created a custom makefile that allows any of my code to
compile directly on the microcontroller. 
I'll bring along the arduino I play around with at home, since I'm pretty sure
I'll have to rewrite the makefile if the Arduino provided in the lab is
different to what I have at home.

\section{Download Sample Code}
The starter sketch is very straightforward:
\begin{lstlisting}
/* Set LED number to an integer. Depending on the model, you'll have
different numbers of these */
int led = 7;
void setup(){
/* Configure pin for Output */
	pinMode(led,OUTPUT);
}
/* OHR: There are two 'void setups' in the manual. Was this a typo? */
void loop(){
/* Set pin 7 value to high, i.e send a current through this pin */
	digitalWrite(led,HIGH);
/* Repeat this forever */
}
\end{lstlisting}

\subsection{What do the following functions do?}
\begin{itemize}
	\item \lstinline{pinMode()}: Configures a specified pin to act as an input
		or output.
	\item \lstinline{loop()}: Equivalant of the 'main' function in the Arduino
		language. All code defined here will repeat over and over as long as the
		device is powered
	\item \lstinline{digitalWrite()}: Given a pin set to OUTPUT, digitalWrite
		will set the current passing through said pin to HIGH or LOW.
	\item \lstinline{delay()}: Pauses the program for a prescribed amount of
		time in milliseconds.
\end{itemize}

\section{Prepare to explore and lab}
\subsection{Calculate the series resistor}
We know that 
$$ I_{max} = \frac{V}{R} $$
$$ 0.02 = \frac{5}{R} $$
Thus,
$$ R = \frac{5}{0.02} = 1\ k\Omega$$

\subsection{Schematic for Potentiometer}
\begin{figure}[H]
	\centering
	\includegraphics[width=0.75\textwidth]{schematic.jpg}
	\caption{Circuit schematic for a potentiometer controlled LED}
\end{figure}

\subsection{Decide on a couple of projects}
Well, one of the things that I've been working on, on my own time, is a motion
sensor that automatically brightens and dims the lights in my room based on
activity.
While, such a project is normally hackathon worthy, i.e can be completed in a 
couple of hours, I'll have to calibrate with Jacob and make sure that we choose
a project that isn't too far beyond each of our skills.
\end{document}
