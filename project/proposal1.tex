%% LyX 2.2.3 created this file.  For more info, see http://www.lyx.org/.
%% Do not edit unless you really know what you are doing.
\documentclass[english]{article}
\usepackage[T1]{fontenc}
\usepackage[latin9]{inputenc}
\usepackage{geometry}
\geometry{verbose,tmargin=1.5in,bmargin=1.5in,lmargin=1.5in,rmargin=1.5in}
\usepackage{color}
\usepackage{babel}
\usepackage{amsmath}
\usepackage{babel}
\newcommand{\GeV}{\,{\rm GeV}}
\newcommand{\MHz}{\,{\rm MHz}}
\usepackage{listings}
\usepackage{float}
\usepackage{graphicx}
\graphicspath{{./plots/}}

\lstdefinestyle{custompy}{
  belowcaptionskip=1\baselineskip,
  breaklines=true,
  frame=L,
  xleftmargin=\parindent,
  language=Python,
  showstringspaces=false,
  basicstyle=\footnotesize\ttfamily,
  keywordstyle=\bfseries,
  commentstyle=\itshape,
  identifierstyle=,
  stringstyle=,
}

\lstset{escapechar=@,style=custompy}


\begin{document}

\title{Project Proposal: Cosmic Ray Detection/Intelligent active suspension}

\author{Omkar H. Ramachandran, Jacob A. Moss}
\maketitle

Note: While both our projects propose the use of a micro-controller, 
\textbf{in both cases, the arduino is only used to amplify the final result
beyond what is already a very challenging project. 
The hero of the show, so to speak, is the analog circuit, with the arduino
only filling minor book-keeping or aesthetic roles}

We have a couple of decent ideas for this project:
\begin{itemize}
	\item \textbf{Muon flux counter}: For this project, we aim to design a 
		simple detector that counts Muons. 
		The physics behind the potential detector is as follows: When a high-
		energy particle hits a luminescent fluid, it scatters photons within
		said fluid.
		With a cleverly placed photo-multiplier tube, we can accurately estimate 
		the intensity of light, thus getting a good idea of a Muon event.
		If we are allowed to use an Arduino, we can further write a counter that
		keeps track of the number of Muons hitting the fluid per unit time,
		thus providing us with the Muon flux.
		We're primarily interested in this project because both of our research
		projects deal primarily with Theoretical Astrophysics/Cosmology, both 
		subjects where a major mode of observation is in the detection of
		high energy particles from cosmic ray scattering
	\item \textbf{Intelligent Active Suspension}: For this project, we propose
		an intelligent active suspension system - tested on a small scale Mechano
		car - that uses a Potentiometer to detect terrain features and dynamically
		adjust the car's ride height.
		Formula 1 cars - specifically the Williams FW 14 - in the early 90s used 
		to have this system, but unlike a general purpose solution of the kind 
		we are proposing, they simply programmed an entire circuit into the car's 
		internal computer, thus selecting the optimum ride height for each corner.
		For our model, we propose a probe that sits by the wheel and is allowed to
		move up and down.
		The position of this probe will then set a potential difference that can
		then be used in conjunction with an appropriate filter to select the ride
		height.
		If we are allowed to use arduinos, we will further make the cars self
		driving, thus enabling it to intelligently detect and navigate out of
		ditches.
\end{itemize}

\end{document}
